\documentclass{article}
\usepackage{amsmath}
\usepackage{amssymb}
\usepackage{mathrsfs}
\usepackage{physics}
\usepackage{comment}
\usepackage{mathtools}
\newcommand{\om}{\omega_n}
\newcommand{\adag}{a^\dagger}
\newcommand\Chi{\mathrm{X}}
\newcommand{\s}{\item[]}
\newtheorem{problem}{Problem}
\title{Master Equation}
\author{Eesh Gupta}
\begin{document}
\maketitle

\section{Background}
\subsection{Closed Quantum System}
A closed system is one in which we consider our system (let's say
an electron) to be the only
entity in the universe. There is no environment or other electrons with which our electron can interact. Then if the initial state of
our electron is \(\ket{\psi(0)}\), then the state of the electron \(\ket{\psi(t)}\)
at any other time \(t\) in the future is completely determined by \(\ket{\psi(0)}\).
In other words, no information about the quantum system ``leaks out'' during
the time evolution. Such deterministic evolution of states (aka unitary time
evolution) allows us to compute
the probabilities of the outcomes of later experiments.
\subsection{Open Quantum System}
In open quantum systems, however, we do recognize that our electron is not the only
entity in the universe. There are other particles and fields that our electron
could interact with. These interactions may result in dissipation of energy and
hence information.

In such quantum zoo, we cannot expect our quantum system to evolve in unitary
fashion. That is, information will ``leak out'' and time evolution will be
a little complicated. Let's say:
\begin{itemize}
  \s Electron : A
  \s Environment : B
  \s State of the electron: \(\rho_A(t)\)
  \s State of the environment: \(\rho_B(t)\)
  \s State of the combined system: \(\rho_{AB}(t)\)
\end{itemize}
Now, the hamiltonian of entire system is
\[H = H_A + H_B + H_{AB}\]
where \(H_A\) only acts on system A, \(H_B\) only acts on system B and \(H_{AB}\) is the
the interaction hamiltonian.

Even though we are taking pains to define the state of the environment, we are
only concerned with the dynamics of our electron or system A.
With this is mind, our objectives in studying open quantum systems is twofold:
\begin{enumerate}
  \item Extract the evolution of \(\rho_A(t)\) from the quantum equation of
  motion of the combined system.
  \[ i\hbar \frac{d }{d t} \rho_{AB}(t) = [H_{AB}, \rho_{AB}(t)]\]
  \item Do so in a way that guarantees the properties of  \(\rho_A(t)\) as
  a density operator (eg. positivity) are preserved in any moment of time.
\end{enumerate}

\section{Approximations}
\subsection{Born-Markov Approximation}
It is worth talking about time scales when discussing correlations between
the system and environment. Suppose that before time \(t = 0\), there exist no
correlations between the electron and the environment. Now at time \(t = 0\)
suppose the electron emits a photon into the environemnt. Now this photon
diappears dramatically fast, and the environment returns quickly to its initial
state. This example shows that the environment is quick to recover from any
internal correlations due to interaction with the system. On the other hand,
the system is relatively slow to recover from these ``perturbations'' or
changes. If the system and environment exchange energy, then it will take system A
a longer time to return back to equilibrium than the environment.

Let \(\tau_B\) be the relaxation time for the environment and \(T_A\) be the
relaxation time for the system. Now suppose we are taking snapshots of changes
to the state of the system \(\Delta \rho_A(t)\) every \(\Delta t\) units of time.
Here we will assume that
\[\tau_B << \Delta t << T_A\]
What does this assumption mean? Suppose the time is \(t =0\) are correlations
are turned on. By the time we take a snapshot, the environment has returned to
its initial equilibrium state. However, the state of the system is still in the
process of ``recovering'' from the correlations.


\end{document}
