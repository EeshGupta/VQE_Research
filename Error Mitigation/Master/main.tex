\documentclass{article}
\usepackage{amsmath}
\usepackage{amssymb}
\usepackage{mathrsfs}
\usepackage{physics}
\usepackage{comment}
\usepackage{mathtools}
\newcommand{\om}{\omega_n}
\newcommand{\adag}{a^\dagger}
\newcommand\Chi{\mathrm{X}}
\newtheorem{problem}{Problem}
\title{Error Mitigation}
\author{Eesh Gupta }
\date{Current Version: March 19, 2020}
\begin{document}
\maketitle
\tableofcontents
\newpage
\section{Introduction}
Isolating physical quantum systems is hard. For this reason, most quantum
systems are considered \textit{open} in the sense that the system
inevitably interacts with its external environment. To illustrate, consider
a qubit represented by 2 discrete states of an electron. Being a charged
particle, that electron is naturally bound to interact with other charged
particles and surrounding electromagnetic fields. This means that the
\textit{environment} can affect the state of the electron as well as the
qubit it represents. These uncontrollable effects are called \textit{noise}
and they can significantly affect our chemistry computations. To understand
the nature of such quantum noise, we will first turn our attentention to
classical noise.
\section{Classical Noise}
Imagine a bit inside a computer hard-drive with external magnetic fields. These
magnetic fields have potential to change the state of the bit with a probability
of let' say \(p\). Let \(i_0, i_1\) be initial probabilities of the bit being
in state 0 and 1 respectively. Similarly, let \(j_1, j_2\) be the new probabilities
after the bit interacts with the magnetic field. Then, we can represent such
\textit{noisy} event as
\[
 \begin{bmatrix}
  j_0 \\
  j_1 \\
 \end{bmatrix}
 =
 \begin{bmatrix}
  1-p &  & p   \\
  p   &  & 1-p \\
 \end{bmatrix}
 \begin{bmatrix}
  i_0 \\
  i_1 \\
 \end{bmatrix}.
\]
\[\begin{bmatrix}
  j_0 \\
  j_1 \\
 \end{bmatrix}
 =
 \begin{bmatrix}
  (1-p)i_0 + pi_1 \\
  pi_0 + (1-p)i_1 \\
 \end{bmatrix}\]
We can make sense of this equation by looking at the equality \(j_0 = (1-p)i_0 + pi_1 \).
Here, \(j_0\) denotes the final probability of the bit being in state 0. If the probability of initial state being 0 is \(i_0\),
then the probability of getting state 0 after interaction is \((1-p)i_0\). In this case,
we multiplied the probability of no-bit-flip  \(1-p\) with \(i_0\). On the other hand, if the probability of initial state being 1
is \(i_1\), then the probability of getting state 0 after interaction is the product of
bit-flip probability \(p\) and \(i_1\). Thus, the probability of getting the final
state of 0 i.e. \(j_0\) is simply the sum of \((1-p)i_0\) and \(pi_1\).
Now, we can write the above equations more succintly as
\[\vec{j} = \hat{E}\vec{i}\]
where \(\hat{E}\) is called the evolution matrix (or the \textit{noise} matrix), and
\(\vec{i}, \vec{j}\) are the initial and final probability distributions respectively.
Then, the final state of the system \(\vec{j}\) is said to be ``linearly'' related to the
initial state of the system \(\vec{i}\).

Note that for the \textit{noise} matrix to describe such a linear transformation, it has
to abide by 2 rules:
\begin{itemize}
  \item \textit{Positivity}: All entries of \(E\) must be non-negative. If \(E\) has
  negative entries, then the vector \(E\vec{q}\) will have negative components i.e.
  negative probabilities. That would be non-sensical.
  \item \textit{Completeness}: The entries in each column of \(E\) must add up to 1.
  Suppose
  \(\vec{i} = \begin{bmatrix}
    x \\
    y \\
  \end{bmatrix}\) and \(
  E = \begin{bmatrix}
    a && c \\
    b && d \\
  \end{bmatrix}\)
  Then
  \[\vec{j} = E\vec{i} = \begin{bmatrix}
    ax + cy \\
    bx + dy \\
  \end{bmatrix}\]
  For \(\vec{i}, \vec{j}\) to be valid probability distribution, their components
  must add up to 1. Then we can assume that \(x+y = 1\), and
  \(ax + cy + bx + dy = 1\) or \((a+b)x + (c+d)y = 1\). For the latter equation
  to be true for any non-trivial inputs \(x, y\) such that \(x+y = 1\),
  both coefficients \(a+b\) and \(c+d\) must be equal to 1. Since these coefficients
  represent the sum of entries of each column of \(E\), we conclude that
  the sum of entries of each column of \(E\) must be equal to 1.
\end{itemize}
\subsection{Markov Process}

Earlier we only looked at one ``noise event''.
Now suppose we have 2 \textit{noisy} gates \(A\) and \(B\). An important
assumption that we can make here is whether gate \(A\) works correctly is
independent of whether gate \(B\) works correctly. That is, the \textit{noisiness}
of gate \(A\) in independent of the \textit{noisiness} of gate B. This assumption
can be \textit{physically reasonable} in cases such as the one where gate \(A\)
and \(B\) are placed a significant distance apart from each other. Then the
process of gate \(A\) and \(B\) being applied in any order is known as Markov
process.

With each noise event being indpendent and being described by a linear transformation,
then the final state after a multiple
noise events is still linearly related to
the initial state.
\section{Quantum Operations}

\subsection{Operator Sum Representation}
In order to describe ``noise events'', we must first develop a mathematical
tools to describe operations on \textit{open} quantum systems. To establish
these tools, we would need some \textit{reasonable} assumptions about the system and the
environment.
\begin{enumerate}
  \item The principal quantum system and environment will be contained a
  \textit{closed} system. A concern that may arise here is that the environment
  has infinite degrees of freedom. However, if our quantum system has \(d\)
  degrees of freedom, then it is safe enough to assume that the environment
  has at most \(d^2\) degrees of freedom.
  \item The principal quantum system \(\rho\) and the environment \(\rho_{env}\)
  start out in a product state. This is generally not true as the quantum system
  and the environment constantly interact with each other. The resultant
  correlations may give rise to various entangled states. As we will find out
  later, the mathematical formalism will work even if the system and the
  environment do not start out in a product state.
  \item
\end{enumerate}

\end{document}
