\documentclass{article}
\usepackage{amsmath}
\usepackage{amssymb}
\usepackage{physics}
\usepackage{comment}
\usepackage{mathtools}
\newcommand{\om}{\omega_n}
\newcommand\Chi{\mathrm{X}}
\title{Quantum Chemistry}
\author{Eesh Gupta }
\date{February 6, 2020}
\begin{document}
\maketitle
\tableofcontents
\newpage
\section{Introduction}
\section{Classical Chemistry}



(What do we mean by spin coordinate \(\omega\)?)
By multiplying a spatial orbital with any one of these spin function, we
obtain spin orbital \(\chi(\textbf{x})\) where \textbf{x} specifies both spatial
coordinate \textbf{r} and spin coordinate \(\omega\).
% \[\chi(\textbf{x_i}) = \psi(\textbf{r})\alpha(\omega) \]
\(\Chi - 5\)
So for every spatial orbital \(\psi(\textbf{r})\), we have 2 spin
orbitals \(\Psi(\mathbf{x_i})\) and \(\Chi(\mathbf{x_i}).\)


\(\bra{\psi(\vec{t})}H\ket{\psi(\vec{t})}\)

\[H\psi = E\psi\]
\end{document}
